\section*{Network cards}

\begin{frame}{Do we have to pay for pf_ring DNA?}{}
	\begin{block}{Feb. 2012 "I don't want money for that."}
		Luca told us that the licenses are for free for people doing research
	\end{block}
	
	\begin{block}{July 2012}
		``We can provide you a few testing licenses, but Silicom manages DNA, not us. We just did the development. 
		\textbf{If you buy cards from them then you won't be (charged) for DNA.}''
	\end{block}
	
	\begin{block}{October 2012 Marco wrote with Luca in CC}
			``For the 10G boards which Jonas is using we can have free licenses for
			PF\_RING only, but DNA (libzero) requires Silicom licenses.''
	\end{block}
	
	\begin{ergo}
		It's pretty clear: We have to pay for licenses. 
		
		\ldots or isn't it?!
	\end{ergo}
\end{frame}

\begin{frame}{Do we have to pay for pf_ring DNA?}{}
	I've posted the news in Siena with the words ``Luca's qote was refering only to
	test licenses - Now we have to pay!!!'
	
		\begin{block}{Jan 2014: Luca found these slides on the web and wrote me}
			``I have told you, and I repeat that once more, that I don’t want money from
			people who do research, but I expect non-research people to pay.''
			
			``Next time you make incorrect public statements make sure that what you state
			is correct. If unsure you better ask.''
			
			``Silicom funded us for some time and in return they asked to buy their
			cards. If you want to buy their competitors cards you have to pay licenses''
	\end{block}
	
	\begin{ergo}
		What's wrong with this guy?
	\end{ergo}

\end{frame}

\begin{frame}{Do we have to pay for pf_ring DNA?}{}
	My interpretation is that he offers licenses for free but would prefer if we
	pay for it.
	
	After trying to clear up misunderstandings he's only sent two messages:
	
	\begin{block}{``I have nothing to do with Silicom anymore''}
			Does that mean that we should not buy Silicom NICs but order the licenses
			directly from ntop?
	\end{block}
	
	\begin{block}{``Just stop using emails from 2012''}
		It seems like he refuses to	coummunicate properly. 
	\end{block}

\end{frame}

\begin{frame}{What now?}{}
	
	\begin{block}{Maybe someone from Pisa should meet him?!}
			\begin{itemize}
				\item Should we buy Silicom or other NICs?
				\item Can we have a discount? Silicom offered 60 licenses for 200 \$ instead
				of 250 \$
			\end{itemize}
	\end{block}
	
	\begin{ergo}
		Let's assume that we have to either buy Silicom NICs (600\$) or Licenses
		(200\$) + Intel cards (450\$)
\end{ergo}
\end{frame}


\section*{PCs}

\begin{frame}{PC purchasement}{}
	No more special condition PCs from Dell
	\begin{block}{But there are still some items available from Dell}
		\begin{itemize}
		  	\item 48 port SFP+ switch for only 7k\euro (with 2 x 40 Gbps uplinks) 			  	
  			\item 16 port KVM switch for about 1,2 k\euro
		\end{itemize}	
	\end{block}
	
\end{frame}

\begin{frame}{Do it yourself PCs}{}
	\begin{block}{A cheap alternative}
		\begin{itemize}
	  		\item Buy mainboard, CPU, RAM, PSU, NIC separately
	  		\item Build ``shelves'' for the racks
	  		\item Put the assembled mainboards on the shelves without any case
	  		\item Use the backdoor fans for cooling
	  			  		 
		\end{itemize}	
	\end{block}
	
	The following prices are based on alternate.de (big discount to be expected!)
\end{frame}

\begin{frame}{Single Desktop CPU}{}
	\begin{block}{Intel 4770K}
		\begin{itemize}
	  		\item CPU: i7-4770K (289 \euro)
	  		\item Mainboard: Z87 (120 \euro)
	  		\item RAM: 32 GB DDR3-1866 (274 \euro)
	  		\item PSU: 450 watts (30 \euro) or 630 watts (60 \euro)
	  		\item HDD: 50\euro
	  		\item NIC: Silicom dual SFP+ (442 \euro)
		\end{itemize}	
		\begin{ergo}
			1155 \euro~without GPU, maybe 1455 with GPU
		\end{ergo}
	\end{block}
\end{frame}

\begin{frame}{Single Server CPU}{}
	\begin{block}{Intel 4930K about 1.3 x 4770k}
		\begin{itemize}
	  		\item CPU: i7-4930K (529 \euro)
	  		\item Mainboard: X97 (225 \euro)
	  		\item RAM: 32 GB DDR3-1866 (274 \euro)
	  		\item PSU: 630 watts (60 \euro)
	  		\item HDD: 50\euro
	  		\item NIC: Silicom dual SFP+ (442 \euro)
		\end{itemize}	
		\begin{ergo}
			1530 \euro~ without GPU, maybe 1800 with GPU
		\end{ergo}
	\end{block}
	Faster but still no ECC or KVM
\end{frame}


\begin{frame}{Dual CPU Server}{}
	\begin{block}{Two Intel E5-2630V2 about 2.0 x 4770k}
		\begin{itemize}
	  		\item CPUs: E5-2630V2 (1138 \euro)
	  		\item Mainboard: Z9PA-D8/iKVM (364 \euro)
	  		\item RAM: 64 GB ECC reg. DDR3-1600 (679 \euro)
	  		\item PSU: 630 watts (60 \euro)
	  		\item HDD: 50\euro
	  		\item NIC: Silicom dual SFP+ (442 \euro)
		\end{itemize}	
		\begin{ergo}
			2703 \euro ~without GPU, maybe 2973 with GPU
		\end{ergo}
	\end{block}
	ECC and KVM included
\end{frame}

\begin{frame}{Dual CPU Server}{}
	\begin{block}{Two Intel E5-2670V2 about 2.6 x 4770k}
		\begin{itemize}
	  		\item CPUs: E5-2670V2 (2958 \euro)
	  		\item Mainboard: Z9PA-D8/iKVM (364 \euro)
	  		\item RAM: 64 GB ECC reg. DDR3-1600 (679 \euro)
	  		\item PSU: 630 watts (60 \euro)
	  		\item HDD: 50\euro
	  		\item NIC: Silicom dual SFP+ (442 \euro)
		\end{itemize}	
		\begin{ergo}
			4503 \euro ~without GPU, maybe 4773 with GPU
		\end{ergo}
	\end{block}
	ECC and KVM included
\end{frame}


\iffalse
\section*{Cryptocoin mining}
\begin{frame}{Make money with GPUs}{}
	\begin{block}{Most probably not feasible but..}
		A single ATI 7970 mines around 150 \euro a month while consuming around 300
		watts
		
		If we'd pay 0.1 \eur~ /kWh it would still be 110 \euro~a month
	\end{block}
	
	\begin{block}{Do it like video rendering companies}
		Let the GPUs mine during idle time: 70\% idle would make 65\euro~a month
		
		60 cards would make about 3900 \euro~ (or 2 new PCs) a month!
	\end{block}
\end{frame}
\fi