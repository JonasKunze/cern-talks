\section{Outlook and conclusion}
\begin{frame}{Current investigations and outlook}{}
	\begin{block}{Implementation of a high performance L1/L2 farm software }
		\begin{itemize}
		  \item Data transmission L0 $\rightarrow$ farm and L2 $\rightarrow$ tapes
		  \item Interface for the actual trigger algorithms
		  \item Monitoring and administration system with a complex web interface
		\end{itemize}
	\end{block}
	
	\begin{block}{Further performance tests}
		\begin{itemize}
		  \item Do we need special network drivers to reduce packet loss?
		  \item How many PCs will be needed?
		\end{itemize}
	\end{block}
\end{frame}

\begin{frame}{Conclusion}{}
	\begin{itemize}
	\item High energy and high precision $\Rightarrow$ a lot of data
	\item Using ordinary ethernet saves money and time and gives you the ability
		to quickly switch between different approaches
	\item Unsteady data production allows new approaches
	 	 \begin{itemize}
		  \item Considering trigger levels as logical object, not as real farms saves
		  a lot of money
		\end{itemize}
	\item The new farm design allows us to have a central software architecture
	which is much easier to implement $\Rightarrow$ Now Mainz will manage the
	whole chain from L0 to persistent memory and install one central farm instead
	of only L2
	\end{itemize}
\end{frame}
