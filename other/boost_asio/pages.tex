\section{Proactor design pattern}
\subsection{Pros and cons}

\lstloadlanguages{C++}
\lstset{language=C++, frameround=fttt, breaklines=true, tabsize=2,
keywordstyle=\color{blue},labelstyle=\tiny, labelstep=5, firstlabel=1, labelsep=5pt, numbers=left}

\begin{frame}{boost::asio}
	\begin{block}{Why I want to use boost::asio}
		\begin{itemize}
		  \item Less code
		  \item More readable code
		  \item Proactor design pattern
		  \item Don't want to reinvent the wheel!
		\end{itemize}
	\end{block}
	\begin{exampleblock}{Proactor design pattern (Think asynchronous!)}
		\begin{itemize}
		  \item Almost no synchronization needed
		  \item Thread per core, not per connection
		  \item Less context switching $\Rightarrow$ better performance
		\end{itemize}
	\end{exampleblock}
\end{frame}

\begin{frame}[fragile]
\frametitle{Sample program}
\framesubtitle{Proactor design pattern: Initiator}
\begin{lstlisting}[frame=trBL,caption={}]{hallowelt}
boost::asio::io_service io_service;
udp::socket socket_(io_service, udp::endpoint(udp::v4(), port_));
socket_.async_receive(boost::asio::buffer(buffer_),boost::bind(&udp_server::handle_receive,this, boost::asio::placeholders::bytes_transferred)); 
for (int i=0; i < 24; i++) 
		boost::thread t(boost::bind(&boost::asio::io_service::run,&io_service));
\end{lstlisting}
\end{frame}

\begin{frame}[fragile]
\frametitle{Sample program}
\framesubtitle{Proactor design pattern: Completion Handler}

\begin{lstlisting}[frame=trBL,caption={}]{hallowelt}
handle_receive(const size_t& bytesReceived) {
	char* data = buffer_;
	buffer_ = malloc(9000);
	socket_.async_receive(boost::...
	DO SOME WORK WITH data HERE!!!
}
\end{lstlisting}
\end{frame}