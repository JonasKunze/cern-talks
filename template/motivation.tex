\section{Motivation}
\subsection{Geschichte der nichtflüchtigen  Datenspeicher}

\begin{frame}{Magnetbänder}{1930 bis heute}
% 35 Terabyte Daten pro Kasette von IBM erzielt:
% http://www.google.com/imgres?imgurl=http://images.derstandard.at/t/12/2010/01/22/1263722120325.jpg&imgrefurl=http://derstandard.at/1263705711220/IBM-erzielt-Rekordspeicherdichte-auf-Magnetband&usg=__2wSWOmccTrKDLYuEScxOAwX5vt8=&h=200&w=300&sz=14&hl=de&start=59&um=1&itbs=1&tbnid=rlSelW5q65C0OM:&tbnh=77&tbnw=116&prev=/images%3Fq%3Dibm%2Bspeicherdichte%26start%3D42%26um%3D1%26hl%3Dde%26sa%3DN%26ndsp%3D21%26tbs%3Disch:1
	\begin{block}{Sequentielles Lesen mit festem Lesekopf}
		\begin{figure}[H]
			\begin{center}
				\includegraphics[width=8cm]{magnetband-lesekopf}
			\end{center}
		\end{figure}
	\end{block}
\end{frame}


% \begin{frame}{Magnetbänder}{1930 bis heute}
% 	\begin{alertblock}{}
%     	\centering
% 			\begin{itemize}
%               \item Langsame Zugriffszeiten
% 			\end{itemize}
%     \end{alertblock}
% 
% 	\begin{alertblock}{Induzierte Spannung abhängig von Fläche eines Bits}
% 		\centering
% 		\begin{itemize}
% 	          \item Hohe Speicherdichte führt zu geringer Induktionsspannung
% 	          \item Schlechtes Signal-Rausch-Verhältnis
% 		 \end{itemize}
%     \end{alertblock}
% \end{frame}

%\begin{frame}{Flash-Speicher}{1994 bis heute (SSD)}
%	\begin{block}{}
%    	\centering
%			\begin{itemize}
%			  \item Persistenz durch elektrische Isolation des Gates
%			  \item Schreiben/Lesen durch Tunneleffekt
%			\end{itemize}
%   \end{block}
%	\begin{figure}[H]
%		\begin{center}
%			\includegraphics[width=5cm]{flash-misfet}
%			\caption{Metall-Isolator-Halbleiter-Feldeffekttransistors}
%		\end{center}
%	\end{figure}
%\end{frame}


\begin{frame}{Festplatten}{1956 bis heute}
	\begin{columns}
	 	\column{.5\textwidth}
			% Lesekopf bei http://de.wikipedia.org/wiki/Perpendicular_Recording nur 10nm von Platte entfernt
			\begin{figure}[H]
				\begin{center}
					\includegraphics[width=4.5cm]{ibm-350}
					\caption{Erste Festplatte: IBM 350 \tiny{\textcolor{gray}{[http://ed-thelen.org/RAMAC/]}}}
				\end{center}
			\end{figure}

	    \column{.5\textwidth}
			\begin{figure}[H]
				\begin{center}
					\includegraphics[width=5cm]{hdd}
					\caption[labelInTOC]{Moderne 750GB Platte \tiny{\textcolor{gray}{[http://de.wikipedia.org/wiki/Festplattenlaufwerk]}}}
				\end{center}
			\end{figure}
	\end{columns}
\end{frame}


\begin{frame}{Festplatten}{1956 bis heute}
	\begin{block}{}
		\begin{itemize}
		  \item Erste Platte ca. 5MB groß, 500kg schwer, 24" Durchmesser, 10kW Leistung, 8,8kB/s, 600ms Zugriffszeit
		  \item Heute 2TB, 3,5"' Durchmesser, unter 6ms Zugriffszeit
		\end{itemize}
	\end{block}
	\begin{alertblock}{Flaschenhals heutiger Computer}<2->
    	\begin{itemize}
          \item Schnelle Zugriffszeiten erfordern hohe Drehzahlen
          \item Hohe Kapazitäten erfordern kleine Speicherbezirke (Induktion nur bis 1994)
        \end{itemize}
    \end{alertblock}
\end{frame}




