\section{Magnetoresistive Effeke}

\subsection{AMR-Effekt}



\begin{frame}{\textbf{A}nisotropic \textbf{M}agneto \textbf{R}esistance}{}

% Der Anisotrope Magnetoresistive Effekt, kurz AMR–Effekt, ist der am längsten bekannte magnetoresistive Effekt und wurde 1857 durch William Thomson, 1. Baron Kelvin entdeckt. Er beruht auf anisotroper (von der Raumrichtung abhängiger) Streuung in ferromagnetischen Metallen. 
%der Wellenfunktionen.1 Der unterschiedliche Widerstand fur den Fall, dass die Stromrichtung
% senkrecht und parallel zur Magnetisierungsrichtung ist, resultiert daraus, dass es aufgrund
% der nicht-kugelsymmetrischen Ladungsverteilung (durch Spin-Bahn-Kopplung verursacht) unterschiedliche Streuquerschnitte fur die
% Leitungselektronen gibt.

%ME_Kapitel3 Seite 6 folgende!!!!


	\begin{block}{elektr. Widerstand abhängig von der Magnetisierungsrichtung}
    	\centering
			\begin{itemize}
              \item 1857 durch Thomson entdeckt
              \item Beruht auf anisotrope Streuung in ferromagnetischen Leitern
              \item Geringer Effekt: ca. 3-4\%
			\end{itemize}
    \end{block}
\end{frame}

\begin{frame}{\textbf{A}\textbf{M}\textbf{R}-Effekt}{Spin-Bahn-Wechselwirkung}
	\begin{block}{Spin-Bahn-Wechselwirkung}
    	\centering
			\begin{itemize}
              \item Keine Kugelsymmetrische Ladungsverteilung
			\end{itemize}
    \end{block}
\begin{figure}[H]
	\begin{center}
		\includegraphics[width=5cm]{amr-orbitale}
		\caption{Grenzflächendarstellung der 3d-Orbitale}
	\end{center}
\end{figure}
\end{frame}

\begin{frame}{\textbf{A}\textbf{M}\textbf{R}-Effekt}{Wirkungsquerschnitt abhängig von der Spin-Ausrichtung}
	\begin{figure}[H]
		\begin{center}
			\includegraphics[width=8cm]{amr-wirkungsquerschnitt}
			\caption{Ausrichtung der Orbitale abhängig vom Spin}
		\end{center}
	\end{figure}
	\begin{align*}
    	R = R_\parallel - \Delta R_\mathrm{max} \cdot \sin^2(\Theta_{JM}) \\
    	\Delta R_\mathrm{max} = R_\parallel - R_\perp
    \end{align*}
\end{frame}


\subsection{GMR-Effekt}
\begin{frame}{\textbf{G}iant \textbf{M}agneto \textbf{R}esistance}{}
%Wird in Strukturen beobachtet, die aus sich abwechselnden magnetischen und nichtmagnetischen dünnen Schichten mit einigen Nanometern
%	      Schichtdicke bestehen

% !!!Die Entdeckung des GMR-Effekts hat zu einem neuen Forschungsgebiet in der Grundlagenforschung der Physik geführt, der Spintronik. Es wurde
% herausgefunden, dass der Spin Einfluss auf die Bewegungsfähigkeit von Elektronen haben kann.


% Ursprüngliche Entdeckung: Kopplung der

% Effekte wie der GMR treten nur in künstlichen Schichtstrukturen auf. Hierbei kommunizieren (koppeln) zwei magnetische Schichten, zum Beispiel aus
% Eisen, über eine Zwischenschicht miteinander. Beim GMR besteht die Mittellage aus einer nicht-ferromagnetischen, stromleitenden Schicht, zum
% Beispiel aus Chrom. Insbesondere diese mittlere Schicht darf nur wenige Nanometer dünn und ihre Grenzflächen müssen von bester Qualität sein.
% Dieser "Sandwich" reagiert empfindlich auf äußere Magnetfelder und verändert dabei stark den elektrischen Widerstand. Genutzt wird der Effekt, um
% beispielsweise Computerfestplatten - und künftig Biochips - über Magnetfeldsensoren auszulesen. In Zukunft werden solche Strukturen im ABS-System
% und an vielen anderen Stellen im Automobilbereich eingesetzt.

% IBM stellte im Dezember 1997 das erste kommerzielle Laufwerk her, das diesen Effekt nutzte. Neben der Anwendung in Festplatten wird der GMR-Effekt
% auch in Magnetfeldsensoren der Automobilindustrie und Automatisierungsindustrie ausgenutzt.
	\begin{block}{}
	    \begin{itemize}
	      \item 1986-88 unabh. von Peter Grünberg und Albert Fert (F) entdeckt (Nobelpreis 2007)
	      \item QM-Effekt basiert auf Spinabhängigkeit der Streuung von Elektronen an Grenzflächen
	    \end{itemize}
    \end{block}
	\begin{exampleblock}{}
    	\begin{itemize}
          \item Stärker als AMR: 80\% bei Zimmertemp. 200\% bei 4,2K
          \item Beruht \textbf{nicht} auf AMR-Effekt
        \end{itemize}
    \end{exampleblock}
\end{frame}



\begin{frame}{\textbf{G}\textbf{M}\textbf{R}-Effekt}{}
	\begin{figure}[H]
		\begin{center}
			\includegraphics[width=8cm]{gmr}
		\end{center}
	\end{figure}
\end{frame}


\begin{frame}{\textbf{G}\textbf{M}\textbf{R}-Effekt}{}
	\begin{figure}[H]
		\begin{center}
			\includegraphics[width=7cm]{gmr-werte}
		\end{center}
	\end{figure}
\end{frame}


\urldef{\gmrurl}{\url}{http://www.techtower.de/subcontent/lab_experiment_gmr.php?zu=2&von=2#ziel}
\begin{frame}{\textbf{G}\textbf{M}\textbf{R}-Effekt}{}
	\begin{figure}[H]
		\begin{center}
			\includegraphics[width=7cm]{gmr-lesekopf}
		\end{center}
	\end{figure}
	\gmrurl
\end{frame}



% \subsection{CMR-Effekt}
% \begin{frame}{\textbf{C}olossal \textbf{M}agneto \textbf{R}esistance}{}
% 	\begin{block}{}
% 	    \begin{itemize}
% 	      \item Entdeckt in den 1950ern 
% 	      \item Vorkommen in gemischvalenten Manganoxiden ($LaMnO_3$) nahe der Curietemperatur
% 	      \item Bei Raumtemperatur ca. 100\% erreicht (oberhalb eines Teslas!)
% 	    \end{itemize}
%     \end{block}
% 	\begin{alertblock}{}
%     	Effekt ist stark von der Temperatur und von der Magnetfeldstärke abhängig
%     	% Also ungeeignet für schwache Felder bei Festplatten
%     \end{alertblock}
% \end{frame}

\subsection{TMR-Effekt}
\begin{frame}{\textbf{T}unnel \textbf{M}agneto \textbf{R}esistance}{magnetische Tunnelwiderstand}
	\begin{block}{}
	    \begin{itemize}
          \item 1975 entdeckt durch Jullière 
          \item Durch GMR-Effekt weitere Untersuchung Anfang der 90er motiviert
	    \end{itemize}
    \end{block}
	\begin{exampleblock}{}
    	\begin{itemize}
          \item 1991: 2,7\% mit amorphem Aluminiumoxid-Isolator
          \item Heute bis zu 600\% bei Raumtemperatur mit CoFeB/MgO/CoFeB (1100\% bei 4,2 K)
        \end{itemize}
    \end{exampleblock}
\end{frame}

\begin{frame}{\textbf{T}\textbf{M}\textbf{R}-Effekt}{}
	\begin{block}{Jullières Ansatz}
    \begin{itemize}
      \item Spin-Erhaltung beim Tunnelprozess
      \item Tunnelleitfähigkeit fur jede Spinrichtung proportional zur effektiven Zustandsdichte der jeweiligen Spinrichtung in beiden Elektroden   
    \end{itemize}
    	
    \end{block}
	\begin{columns}
	 	\column{.5\textwidth}
			\begin{block}{Spinpolarisation}
		 		\begin{align*}
			    	P = \frac{N_\uparrow - N_\downarrow}{N_\uparrow + N_\downarrow}
			    \end{align*}
			\end{block}
	    \column{.5\textwidth}
	    	\begin{block}{TMR nach Jullière}
				\begin{align*}
			    	\frac{R_{ap}-R_{p}}{R_{p}} = \frac{2P_1P_2}{1-P_1P_2}
		    	\end{align*}
	 		\end{block}
	\end{columns}
\end{frame}