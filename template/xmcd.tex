\section{Röntgenabsorptionsspektroskopie}
%\begin{frame}{Örtlich- und zeitlich aufgelöste Messungen erforderlich}{}
%	\begin{block}{}
%		Magnetische Eigenschaften von Übergangsmetallen (Fe, Co und Ni) hauptsächlich durch 3d Valenzelektronen bestimmt \\
%		\begin{ergo}
%			Messe Anzahl von $\uparrow$- und $\downarrow$ 3d-Elektronen (Zustandsdichte)
%		\end{ergo}
%	\end{block}
%\end{frame}

\subsection{XAS}
\begin{frame}{Röntgenabsorptionsspektroskopie XAS}{}
%Im Bändermodell sind das spin up - und das spin down -Band aufgrund der Austauschwechselwirkung aufgespalten und um den Betrag der
% Austauschaufspaltung gegeneinander verschoben. Da alle Zustände bis zur FERMI-Energie besetzt (für $T\to0$) und darüber alle frei sind, gibt es
% mehr spin up - als spin down -Elektronen - daher besitzt die Probe ein magnetisches Moment. 
	\begin{block}{}
		Weiche Röntgenstrahlung zwischen 50eV and 2000eV
	\end{block}

	\begin{figure}[H]
		\begin{center}
			\includegraphics[width=8cm]{xas}
			\caption{Absorptionsspektrum einer Cu-Fe-Ni Probe \tiny{\textcolor{gray}{[Stan]}}}
		\end{center}
	\end{figure}
\end{frame}

\begin{frame}{Röntgenabsorptionsspektroskopie XAS}{}
	\begin{figure}[H]
		\begin{center}
			\includegraphics[width=11cm]{l2l3-kanten}
			\caption{Termschema und Absorptionsspektrum von Eisen}
		\end{center}
	\end{figure}
\end{frame}

% \begin{frame}{Magnetischer Dichroismus}{}
% 	%Durch das anregende rechtszirkular polarisierte Photon
% 	% wird ein spin down -Elektron in das unbesetzte spin down -Band oberhalb der FERMI-Energie angehoben, während ein linkszirkular polarisiertes Photon
% 	% ein spin up -Elektron in das unbesetzte spin up -Band anregen würde. Da im spin down -Band mehr freie Plätze sind als im spin up -Band, ist nach
% 	% FERMIS goldener Regel auch der Absorptionskoeffizient von rechtszirkular polarisierter Strahlung größer als von linkszirkular polarisierter - die
% 	% Absorption hängt von der Helizität des Photons (seinem Spin) ab.
% 	
% 	% Der Dichroismus bekam seinen Namen (,,Zweifarbigkeit``) vom der unterschiedlichen Farbe, mit der das Medium in Transmission erscheint, wenn man es
% 	% mit linear polarisiertem polychromatischem Licht bestrahlt. Der unterschiedliche Farbeindruck wird von der Abhängigkeit der Dispersionskurven von
% 	% Polarisationszustand des Lichtes hervorgerufen. 
% 		\begin{block}{Dichroismus}
% 	    	Abhängigkeit der Lichtabsorption von der Polarisation der einstrahlenden elmag. Welle (hier links- oder rechtszirkular) \\
% 	    \end{block}
% 	
% 	 % Hochenergetische Variante des magneto-optischen Kerreffektes
% 	 % magneto-optischer Kerreffekt Interessant für magneto-optischen Datenträgern (Mini Disc) 
% 	 % Der magnetooptische Kerr-Effekt sollte nicht mit dem elektrooptischen Kerr-Effekt  verwechselt werden, bei dem elektrische Felder die
% 	 % Polarisationsebene drehen. Der magnetooptische Kerr-Effekt wurde 1876 von John Kerr entdeckt.
% 	 \begin{figure}[H]
% 		\begin{center}
% 			\includegraphics[width=7cm]{dichroism}
% 			\caption{Zirkulardichroismus von Eisen \tiny{\textcolor{gray}{[Eim]}}}
% 		\end{center}
% 	\end{figure}
% \end{frame}


\subsection{XMCD}


\begin{frame}{\textbf{X}-ray \textbf{M}agnetic \textbf{C}ircular \textbf{D}ichroism}{}
 %Experimenteller beweis für drehimpuls: 1936 durch Richard A. Beth	
 	\begin{block}{Dichroismus}
    	Abhängigkeit der Lichtabsorption von der Polarisation der einstrahlenden elmag. Welle (hier links- oder rechtszirkular) \\
    \end{block}

	\begin{block}{Zirkular polarisiertes Licht hat einen Bahndrehimpuls!}<2->
		Bahndrehimpuls wird an Elektron übertragen (Dipolübergang):
		\begin{align*}
			\Delta l&= \pm 1	&	\Delta j &= 0, \pm 1 \\
			\Delta m_l &= m_{\gamma}	&	\Delta m_j &= m_{\gamma}	& \Delta m_s=0
		\end{align*}
	\end{block}
\end{frame}


\begin{frame}{\textbf{X}-ray \textbf{M}agnetic \textbf{C}ircular \textbf{D}ichroism}{}
 	\setbeamercovered{invisible}
		\begin{figure}[H]
			\begin{center}
				\includegraphics[width=9cm]{xmcd-termschema1}<1>
				\includegraphics[width=9cm]{xmcd-termschema2}<2>
				\includegraphics[width=9cm]{xmcd-termschema3}<3>
				\includegraphics[width=9cm]{xmcd-termschema4}<4>
			\end{center}
		\end{figure}
	\setbeamercovered{transparent}
\end{frame}

\begin{frame}{\textbf{X}-ray \textbf{M}agnetic \textbf{C}ircular \textbf{D}ichroism}{}
	\begin{figure}[H]
		\begin{center}
			\includegraphics[height=7cm]{xmcd-theo}
		\end{center}
	\end{figure}
\end{frame}

% in Halbleiter-Kristallen beobachtet, dass sich Spin-Zustände über makroskopische Längen von etwa 0.1 mm kohärent bewegen können.


% Its foremost strengths are the element-specific, quantitative separation and determination of spin and orbital magnetic moments and their
% an-isotropies

%L-edge absorption studies (2p to 3d transitions)


% Untersucht werden (L3,L2) - Übergänge von 2p-Zuständen in die austauschaufgespaltenen 3d-
% Zustände, welche für den Magnetismus der Übergangsmetalle verantwortlich sind. Aufgrund
% der Spin-Bahn-Wechselwirkung ist das 2p-Nivau in die Zustände 2p1/2 und 2p3/2 aufgespalten.
% Damit gibt es für die Quantenzahl mj für das 2p1/2-Niveau zwei Werte, nämlich +1/2 und -1/2
% und für das 2p3/2–Niveau außer diesen beiden Werten noch die Werte +3/2 und -3/2. Damit
% besitzt das 2p3/2–Niveau vier Zustände, während das das 2p1/2-Niveau nur zwei Zustände
% besitzt. Die erwartete Absorptionsausbeute für das 2p3/2 –Niveau ist also höher als für das
% andere. Erwarten würde man ein Signal doppelter Intensität. Da aber noch andere Faktoren,
% z.B. die Energieauflösung der Apparatur, eine Rolle spielen, sieht der reale Verlauf etwas
% anders aus.
% 12
% Die für die Übergänge vom 2p3/2 –Niveau ins 3d-Band charakteristische Kante im
% Absorptionsspektrum wird als L3-Kante, entsprechend die des 2p1/2-Niveau als L2-Kante
% bezeichnet.


\begin{frame}{XMCD}{}
	\begin{figure}[H]
		\begin{center}
			\includegraphics[width=9cm]{xmcd}
		\end{center}
	\end{figure}
\end{frame}


\subsection{Röntgenoptik}
\begin{frame}{Hoher Anspruch an die Röntgenstrahlung}{}
	\begin{block}{Elementselektive Mikroskopie}
		\begin{itemize}
		  \item Durchstimmbare Photonenenergie (Bereich ca. 50eV bis 2000eV)
		  \item Extrem hohe Intensität nötig
		\end{itemize}		
	\end{block}

	\setbeamercovered{invisible}
		\begin{columns}<2->
		 	\column{.4\textwidth}
		 	\begin{figure}[H]
				\begin{exampleblock}{Synchrotronstrahlung!}
					Es gibt z.Zt. 44 Synchrotronlaboratorien Weltweit, 5 in Deutschland
				\end{exampleblock}
			\end{figure}
		    \column{.6\textwidth}
			\begin{figure}[H]
				\begin{center}
					\includegraphics[width=\textwidth]{synchrotrondichte}
					\caption{Globale Synchrotrondichte  \tiny{\textcolor{gray}{[Schu]}}}
				\end{center}
			\end{figure}
		\end{columns}
	\setbeamercovered{transparent}
\end{frame}

\begin{frame}{Zirkular Polarisierte Röntgenstrahlung}{}
	\begin{exampleblock}{Helikale Undulatoren im Synchrotron}
		Erreichte Intensität ca. $10^8$ mal größer als Röntgenröhren
	\end{exampleblock}
	\begin{figure}[H]
		\begin{center}
			\includegraphics[width=\textwidth]{undulator}
		\end{center}
	\end{figure}
\end{frame}

\begin{frame}{Röntgen-Linsen}{}
% 		Der äußerste Ring kann trotz des hohen Aspektverhältnisses elektronenlithographisch mit einer Wandstärke von 20 nm hergestellt werden kann und
% entspricht der Auflösung des Systems.
	\begin{block}{Fresnel-Zonenplättchen}
		Konzentrische, nach außen bis auf 20 nm dünner werdende Zylinder\\
	\end{block}
	\begin{figure}[H]
		\begin{center}
        %(c) http://www.helmholtz-berlin.de/forschung/grossgeraete/mikroskopie/arbeitsgebiete/entwicklung-von-beugender-optik/index_en.html
			\includegraphics[width=8cm]{fresnel}
			\caption{Querschnitt eines Zonenplättchens  \tiny{\textcolor{gray}{Foto: [Eim]}}}
		\end{center}
	\end{figure}
\end{frame}

\begin{frame}{Fresnel-Zonenplättchen}{}
	% http://x-ray-optics.eu/index.php?option=com_content&view=article&id=22&Itemid=125&lang=de
	\begin{figure}[H]
		\begin{center}
			\includegraphics[width=10cm]{fresnel-herleitung}
		\end{center}
	\end{figure}

	\begin{columns}
	 	\column{.5\textwidth}
			\begin{block}{}<2->
				\begin{align*}
					\Delta_1 &= \sqrt{g^2+r_n^2}-g \\
					\Delta_2 &= \sqrt{b^2+r_n^2}-b
				\end{align*}
			\end{block}
	    \column{.5\textwidth}
			\begin{block}{Konstruktive Interferenz}<3->
				\begin{align*}
					\Delta_{total} &= \Delta_1+\Delta_2 = \frac{n\lambda}{2}\\
				\end{align*}
			\end{block}
	\end{columns}
\end{frame}

\begin{frame}{Fresnel-Zonenplättchen}{}
	\begin{block}{}
		\begin{align*}
			&\Delta_{total} = \Delta_1+\Delta_2 = \frac{n\lambda}{2}\\
			&\Rightarrow \sqrt{g^2+r_n^2}-g+ \sqrt{b^2+r_n^2}-b = \frac{n\lambda}{2} \\
			&\Rightarrow r_n^2 = \frac{\frac{n^4\lambda^4}{16}+\frac{n^3\lambda^3(g+b)}{2}+n^2\lambda^2(gb+(g+b)^2)+4n\lambda gb (g+b)}{4(g+b)(n\lambda
					+(g+b))+n^2\lambda^2}
		\end{align*}
	\end{block}

	\begin{columns}
	 	\column{.5\textwidth}
			\begin{block}{$\lambda \ll g,b$}<2->
				\begin{align*}
					r_n^2 &\approx n\lambda \frac{gb}{g+b} = n\lambda f\\
					mit\quad \frac{1}{f} &= \frac{1}{g}+\frac{1}{b}					
				\end{align*}
			\end{block}
	    \column{.5\textwidth}
			\begin{block}{Ringbreite}<3->
				\begin{align*}
					dr_n = \frac{\partial r_n}{\partial n} = \frac{r_n}{n}
				\end{align*}
			\end{block}
	\end{columns}
\end{frame}

\begin{frame}{Fresnel-Zonenplättchen}{}
	\begin{figure}[H]
		\begin{center}
			\includegraphics[width=9cm]{fresnel-ordnungen}
		\end{center}
	\end{figure}
\end{frame}

\begin{frame}{Fresnel-Zonenplättchen}{Ortsauflösung}
	\begin{block}{Rayleigh-Kriterium: maximale Ortsauflösung}
    	
    	\begin{columns}
		 	\column{.4\textwidth}
		    	\begin{figure}[H]
						\includegraphics[width=3cm]{rayleigh}
				\end{figure}
		    \column{.6\textwidth}
		    	Für eine monochromatische Einstrahlung gilt: 
				\begin{align*}
					\delta = 1,22 \frac{dr_N}{m}
				\end{align*}

% 				\begin{align*}
% 					\delta = 1,22\frac{\lambda}{2 NA} = 1,22 \frac{dr_N}{m}
% 				\end{align*}
		\end{columns}
	\end{block}
% 	\begin{block}{Abbe-Theorie: Numerische Apertur}<2->
%     %Die numerische Apertur (Formelzeichen NA, n.A. oder AN) beschreibt das Vermögen eines optischen Elements, Licht zu fokussieren. Bei Objektiven
%     % bestimmt sie die minimale Größe des in seinem Fokus erzeugbaren Lichtflecks. Insbesondere ist die NA eine das Auflösungsvermögen wesentlich
%     % bestimmende Größe. Genauer ergibt sich die numerische Apertur aus dem Produkt des Sinus des halben objektseitigen Öffnungswinkels
%     % (Akzeptanzwinkel) $\Theta_E  und dem Brechungsindex n des Immersionsmediums (Material zwischen Objektiv und Fokus):
% 		Kenngröße für die "`Fokussierbarkeit"' eines optischen Elements 
% 		\begin{align*}
% 			NA = n \cdot sin(\Theta_O) \approx 1\cdot tan(\Theta_O) = \frac{r_N}{f} = \frac{m\lambda}{2 \cdot dr_N}
% 		\end{align*}
% 	\end{block}
\end{frame}

\begin{frame}{Fresnel-Zonenplättchen}{Ortsauflösung}
	\begin{figure}[H]
		\begin{center}
			\includegraphics[width=5cm]{fresnel-foto-aussen}
			\caption{Ni Zonenplättchen Uni Göttingen 2000 ($dr_N=22nm$)  \tiny{\textcolor{gray}{[Eim]}}}
		\end{center}
	\end{figure}
\end{frame}

\subsection{MTXM}
\begin{frame}{Magnetische Röntgentransmissionsmikroskopie}{MTXM}
	\begin{figure}[H]
		\begin{center}
			\includegraphics[width=11cm]{mtxm}
		\end{center}
	\end{figure}
\end{frame}

\begin{frame}{Magnetische Röntgentransmissionsmikroskopie}{MTXM}
	\begin{block}{Erste Bilder am Bessy I 1996: }
		\begin{figure}[H]
			\begin{center}
            	\includegraphics[width=5cm]{mtxm-linkszirkular}
				\includegraphics[width=5cm]{mtxm-rechtszirkular}
			\end{center}
			\caption{Erste MTXM Aufnahmen von $Fe_{72} Gd_{28}$ (links und rechts-polarisierte Einstrahlung) \tiny{\textcolor{gray}{[Fis]}}}
		\end{figure}
	\end{block}
\end{frame}

\begin{frame}{Magnetische Röntgentransmissionsmikroskopie}{MTXM}
\begin{figure}[H]
	\begin{center}
		\includegraphics[width=6cm]{mammos}
		\caption{\textbf{M}agnetic \textbf{A}mplifying \textbf{M}agneto-\textbf{O}ptical \textbf{S}ystem \tiny{\textcolor{gray}{[Schu]}}}
	\end{center}
\end{figure}
\end{frame}

\subsection{STXM}
\begin{frame}{Rasterndes Verfahren}{STXM}
	\begin{block}{\textbf{S}canning \textbf{T}ransmission \textbf{X}-ray \textbf{M}icroscope}
		\begin{figure}[H]
			\begin{center}
				\includegraphics[width=10cm]{stxm}
			\end{center}
		\end{figure}
	\end{block}
\end{frame}


\begin{frame}{STXM}{Vortice}
	\begin{columns}
	 	\column{.5\textwidth}
			\begin{block}{Advanced Light Source in Berkeley}<2->
				2 Elektronen-bunches: 70ps breit, 328ns Abstand
			\end{block}
			\begin{block}{MPI Stuttgart 2004}<3->
				$400ps$ Strompuls in $20 ps$-Schritten vor Röntgenpuls
			\end{block}
	    \column{.5\textwidth}
			\begin{figure}[H]
				\begin{center}
					\includegraphics[width=5cm]{stxm-microcoil}
					\caption{Mikrospule um eine $16\mu m^2$ ferromagn. Schicht auf $100 nm ~ Si_3N_4$ Membran
					\tiny{\textcolor{gray}{[Stoll]}}}
				\end{center}
			\end{figure}	    
	\end{columns}
\end{frame}

\begin{frame}{STXM}{Vortice}
	\setbeamercovered{invisible}
		\begin{columns}
		 	\column{.5\textwidth}
				\begin{figure}[H]
					\begin{center}
						\includegraphics[width=5cm]{stxm-vortex}
						\caption{Simulierte (erste Reihe) und gemessene Magnetisierung \tiny{\textcolor{gray}{[Stoll]}}}
					\end{center}
				\end{figure}
		    \column{.5\textwidth}<2->
				\begin{figure}[H]
					\begin{center}
						\includegraphics[width=3cm]{vortex}
						\caption{Vortice als neues Speichermedium?}
					\end{center}
				\end{figure}	    
		\end{columns}
	\setbeamercovered{transparent}
	\begin{block}{}<3->
		\tiny Vidoes (H. Stoll):  http://www.nature.com/nature/journal/v444/n7118/suppinfo/nature05240.html
	\end{block}
\end{frame}

\begin{frame}{Zusammenfassung}{}
	\begin{block}{Magnetische Transmissionsröntgenmikroskopie}
		\begin{itemize}
			\pro Hohe laterale Auflösung: $20nm$ bereits erreicht
			\pro Elementselektivität
			\pro Hohe zeitliche Auflösung (reversible Prozesse)
			\contra Hoher Anspruch an die Probe
			\contra Synchrotronstrahlung benötigt
			\contra Nicht für senkrecht zum Strahl magnetisierte Proben geeignet
		\end{itemize}
	\end{block}
\end{frame}


% 
% \section{Alternative Verfahren}
% 
% \subsection{XMCD-PEEM}
% \begin{frame}{Photoemissionsmikroskopie}{XMCD-PEEM}
% 	\begin{block}{Photoelektronen durch zirkular polarisierte Röntgenstrahlen}
% 		Polarisationsabhängige Auslösewahrscheinlichkeit\\
% 		\begin{ergo}
% 	    	Elektronenmikroskop zur Detektion
% 	    \end{ergo} 
% 	\end{block}
% 
% 	\begin{figure}[H]
% 		\begin{center}
% 			\includegraphics[width=5cm]{peem}
% 		\end{center}
% 	\end{figure}
% \end{frame}
% 
% \begin{frame}{Photoemissionsmikroskopie}{XMCD-PEEM}
% 	\begin{block}{}
%     %Geringe Austrittstiefe der Sekundärelektronen
% 		\begin{itemize}
% 		  \item Gut geeignet für Untersuchungen von Oberflächen
% 		  \item Optimal für parallel zur Oberfläche magnetisierte Schichten
% 		  \item Auflösung bis $3nm$ erreicht %http://unicorn.mcmaster.ca/highlights/capers/capers.html
% 		\end{itemize}
% 	\end{block}
% \begin{figure}[H]
% 	\begin{center}
% 		\includegraphics[height=4cm]{peem-example}
% 		\caption{Pb bedampftes Si \tiny{\textcolor{gray}{[Mcm]}}}
% 	\end{center}
% \end{figure}
% \end{frame}
% 
% \subsection{MFM}
% \begin{frame}{Magnetkraftmikroskopie (MFM)}{}
% %-Auflösung: ca. 50 nm
% %-Schlechte/Keine Zeitauflösung!
% %+Keine teure Synchrotronstrahlung nötig
% 	\begin{block}{Spezielle Rasterkraftmikroskopie (50nm Auflösung erreicht)}
%    		ferromagnetischer Abtastkopf, zwei Durchläufe pro Bildzeile:
%     	\begin{itemize}
% 		  \item 1: Aufnahme des Höhenprofils
% 		  \item 2: Feldstärkenmessung (konstanter Abstand $<$ 100nm)
% 		\end{itemize}
% 	\end{block}
% 	\begin{figure}[H]
% 		\begin{center}
% 			\includegraphics[width=7cm]{mfm-tape}
% 			\caption{Topographische und magnetische Karte eines Magnetbandes \tiny{\textcolor{gray}{[Sci]}}}
% 		\end{center}
% 	\end{figure}
% \end{frame}
% 
% \subsection{Kerr-Mikroskopie}
% \begin{frame}{Kerr-Mikroskopie}{}
% 	\begin{block}{Magnetooptischer Kerreffekt}
% 		Magnetisierung beeinflusst Polarisationszustand der Reflexion
% 		\begin{itemize}
% 		  \item Linear polarisierte Einstrahlung
% 		  \item Elliptisch polarisierte Reflexion (Kerrwinkel) 
% 		\end{itemize}
% 	\end{block}
% 
% 	\begin{block}{Reflexion von Licht im optischen Bereich}
% 		\begin{itemize}
% 		  \item Geringe Auflösung von ca. $1\mu m$
% 		  \item Nur Oberflächen betrachtbar
% 		\end{itemize}
% 	\end{block}
% \end{frame}
% 
% \begin{frame}{Kerr-Mikroskopie}{}
% 	\begin{figure}[H]
% 		\begin{center}
% 			\includegraphics[width=10cm]{kerr-mikroskopie}
% 			\caption{Magnetische Domänen einer Fe-Ni-Oberfläche  \tiny{\textcolor{gray}{[Kerr]}}}
% 		\end{center}
% 	\end{figure}
% \end{frame}