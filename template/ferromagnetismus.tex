\section{Ferromagnetismus}

\begin{frame}{Ferromagnetismus}{Kollektives Phänomen}
	\begin{block}{Mehrheitlich parallele Ausrichtung der Elektronenspins}
		Parallelstellung zunächst ungünstig, da nach Pauli höhere Energieniveaus besetzt werden müssen
	\end{block}

	\begin{block}{Austauschwechselwirkung verringert potentielle Energie}<2->
		Bei wenigen Festkörpern ist parallele Ausrichtung energetisch günstiger
	\end{block}
%	\setbeamercovered{invisible}
%		\begin{figure}
%			\begin{center}
%				\includegraphics[width=8cm]{ferromagnetismus}<3->
%			\end{center}
%		\end{figure}
%	\setbeamercovered{transparent}
\end{frame}

\begin{frame}{Zustandsdichte von Ferromagneten}{}
	\begin{figure}[H]
		\begin{center}
			\includegraphics[width=10cm]{ferromagnetismus-zustandsdichte}
		\end{center}
	\end{figure}
\end{frame}