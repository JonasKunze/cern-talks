\appendix % Anang

\begin{frame}{Kernspeicher}{1950 bis Ende 70er}

% Ferrit ist die Bezeichnung für kristallographische Modifikationen des Eisens, die ein kubisch-raumzentriertes  Kristallgitter bilden.
%Ferrite sind elektrisch schlecht oder nicht leitende ferrimagnetische keramische Werkstoffe  aus Eisenoxid Hämatit  (Fe2O3), seltener Magnetit 
% (Fe3O4) und weiteren Metalloxiden. Ferrite  leiten im nicht gesättigten Fall den magnetischen Fluss sehr gut und haben eine hohe magnetische
% Leitfähigkeit (Permeabilität). Diese Werkstoffe weisen somit im Regelfall einen kleinen magnetischen Widerstand auf.
	\begin{figure}[H]
		\begin{center}
			\includegraphics[width=9cm]{kernspeicher}
		\end{center}
	\caption{Speicherung in Ferritkernen \tiny{\textcolor{gray}{[http://de.wikipedia.org/wiki/Kernspeicher]}}}
	\end{figure}
\end{frame}

\begin{frame}{Kernspeicher}{1950 bis Ende 70er}
	\begin{block}{Schreibvorgang}
    	\centering
		\begin{itemize}
		  \item Strom $I_K$ benötig für Koerzitivfeldstärke
		  \item $I_K$ auf Zeilen- und Spaltendraht verteilt
		\end{itemize}
    \end{block}

	\begin{block}{Lesevorgang: destruktiv, wahlfrei}<2->
    	Spannungsimpuls beim schreiben in Leseleitung induziert:
		\begin{description}
		  \item[Hoher Impuls:] Magnetisierung wurde umgepolt
		  \item[Kleiner Impuls:] Keine Umpolung stattgefunden
		\end{description}
    \end{block}
\end{frame}